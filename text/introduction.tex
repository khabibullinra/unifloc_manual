
\addcontentsline{toc}{chapter}{Введение}    % Добавляем его в оглавление

Документ описывает набор макросов и функций \unf{} для проведения инженерных расчетов систем нефтедобычи в Excel. Макросы \unf{} позволяют строить расчетные модули, которые могут быть использованы для изучения математических моделей систем нефтедобычи и развития навыков проведения инженерных расчётов, изучения нефтяного инжиниринга и проведения расчетов.

Макросы и функции \unf{} охватывают основные элементы математических моделей систем <<пласт - скважина - скважинное оборудование>> - модель физико-химических свойств пластовых флюидов (PVT модель), модели многофазного потока в трубах, скважинном оборудовании, пласте, модели скважин и узлового анализа систем нефтедобычи.  

Для использования \unf{} требуются навыки уверенного пользователя MS Excel, желательно знание основ программирования и теории добычи нефти. 

Алгоритмы реализованные в  \unf{} не претендуют на полноту и достоверность и ориентированы на учебные задачи и проведение простых расчётов. Руководство пользователя также не претендует на полноту описания системы (часто получается, что описание отстаёт от текущего состояния \unf{}). Все приводится как есть. Более надёжным способом получения достоверной информации о работе макросов \unf{} является изучение непосредственно расчётного кода в редакторе VBE.


\url{https://github.com/unifloc/unifloc_vba}

Хабибуллин Ринат (khabibullin.ra@gubkin.ru)  