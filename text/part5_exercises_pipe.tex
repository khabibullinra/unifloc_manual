\section{\mintinline{vb.net}{ex051.MF_pipeline.xlsm} - Расчет распределения давления в трубе}
Расчет многофазных потоков в трубе - ключевой для анализа работы скважин и скважинного оборудования. Под расчетом трубы подразумевается в первую очередь расчет распределения давления. Иногда требуется рассчитать и распределение температуры. 
На распределение давления в трубе среди прочих параметров влияют режим потока газожидкостной смеси и явление проскальзывание газа. Расчет проводится с использованием многофазных корреляций. 

\subsection{Упражнение}
Упражнение показывает расчет потока через трубопровод/скважину со сложной траекторией. Расчет может проводится в нескольких вариантах относительно потока. 

\begin{figure}[h!]
	\center{\includegraphics[width=1\linewidth]{Ex51_1}}
	\caption{Упражнение \mintinline{vb.net}{ex050.MF_pipeline.xlsx} со всеми заполненными полями }
	\label{ris:Ex51_1}
\end{figure}

Выполните следующие задания
\begin{enumerate}
	\item Постройте зависимость распределение давления в трубе снизу вверх и сверху вниз
	 
	\item Попробуйте подобрать параметры расчета так, чтобы кривые расчета в разных направлениях совпали
	
	\item Рассчитайте параметры для согласования расчетов с использованием функции калибровки расчета
	
\end{enumerate}

Пример также показывает варианты задания данных по трубопроводу и возможность расширенного вывода результатов.

Для выполнения расчетов используйте следующие функции \unf{}:
\begin{itemize}
	
	\item \mintinline{vb.net}{MF_p_pipeline_atma}
	\item \mintinline{vb.net}{MF_calibr_pipeline}
\end{itemize}

Убедитесь, что для всех построенных графиков вы понимаете их поведение при изменении давления и температуры. 

\subsection{Вопросы для самоконтроля}
Для самоконтроля ответьте на следующие вопросы:

\begin{enumerate}
	
	\item Какие параметры влияют на перепад давления в трубе?
	\item В чем отличие различных температурных моделей расчета?
	\item Может ли в трубопроводе давление ниже по потоку (на выходе) быть больше чем выше по потоку (на входе)?
	\item Насколько сильно влияет на расчет выбор гидравлической корреляции? Сравните насколько должна быть велика погрешность в исходных данных для того чтобы пренебречь выбором корреляции?
	\item Насколько сильно влияет на расчет температура?
					 
	
\end{enumerate}

\subsection{Дополнительные вопросы и задания}

Для того, чтобы глубже разобраться в расчете свойств флюидов с использованием \unf{} ответьте на дополнительные вопросы, которые легко превращаются в задания.

\begin{enumerate}
	
	\item Постройте распределение давления в скважине с фонтанным режимом работы. Учтите длину НКТ, наличие штуцера и приток из пласта.
	\item Постройте график распределения давления в затрубном пространстве при известном давлении на приеме насоса и затрубном давлении. Найдите значение динамического уровня.

\end{enumerate}


\section{\mintinline{vb.net}{ex052.MF_pipeline_hdyn.xlsm} - Расчет динамического уровня}
Пример и упражнение показывает два режима расчета потока - барботаж - поток газа через неподвижный столб жидкости и расчет распределения давления в столбе газа. Также применяются функции для работы с кривыми заданными таблично.

\subsection{Упражнение}
Моделируется поток в межтрубном пространстве.  Задано давление на приеме насоса, конструкция скважины и затрубное давление. Требуется построить кривую распределения давления в затрубе с учетом наличии в скважине динамического уровня.

\begin{figure}[h!]
	\center{\includegraphics[width=1\linewidth]{Ex52_1}}
	\caption{Упражнение \mintinline{vb.net}{ex052.MF_pipeline_hdyn.xlsm} со всеми заполненными полями }
	\label{ris:Ex52_1}
\end{figure}

Выполните следующие задания
\begin{enumerate}
	\item Постройте зависимость распределение давления в трубе снизу вверх при нулевом дебите жидкости. Опция \mintinline{vb.net}{znlf=True}
	
	\item Постройте зависимость распределения давления в трубе сверху вниз для столба газа
	
	\item Оцените величину динамического уровня - уровня на котором пересекаются кривые распределения давления - как точку пересечения построенных кривых. 
	
	\item Постройте зависимость динамического уровня от расхода газа. 
	
\end{enumerate}

Построить кривую зависимости динамического уровня от расхода газа можно с использование макроса, которые выполнит основной расчет несколько раз и в нужном месте соберет исходные данные и результаты.

Для выполнения расчетов используйте следующие функции \unf{}:
\begin{itemize}
	
	\item \mintinline{vb.net}{MF_p_pipeline_atma}
	\item \mintinline{vb.net}{crv_intersection}
\end{itemize}

Убедитесь, что для всех построенных графиков вы понимаете их поведение при изменении давления и температуры. 

\subsection{Вопросы для самоконтроля}
Для самоконтроля ответьте на следующие вопросы:

\begin{enumerate}
	
	\item Как будет менять величина динамического уровня при изменении расхода газа в затрубе?
	\item Какие параметры будут влиять на величину динамического уровня?
	\item Всегда ли можно найти значение динамического уровня?
	
	
\end{enumerate}

\subsection{Дополнительные вопросы и задания}

Для того, чтобы глубже разобраться в расчете динамического уровня с использованием \unf{} ответьте на дополнительные вопросы, которые легко превращаются в задания.

\begin{enumerate}
	
	\item Для скважины, где давление на приеме может меняться как будет вести себя динамический уровень?.
	\item Возможна ли такая ситуация, что при остановке скважины динамический уровень будет увеличиваться? Можно ли ее смоделировать?
	
\end{enumerate}

