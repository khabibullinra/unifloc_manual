\section{Расчет коэффициентов сепарации}

Процессы сепарации на приеме погружного оборудования значительно влияют на процесс добычи. Как при естественной, так и при искусственной сепарации (при применении газосепараторов) меняются свойства многофазного потока, уменьшается газлифтный эффект, изменяется режим работы центробежного насоса.

В данном упражнении помимо стандартного определения PVT свойств требуется задать термобарические условия на приеме погружного оборудования (в месте, где происходит сепарация) и конструктивные параметры


\begin{figure}[h!]
	\center{\includegraphics[width=1\linewidth]{Ex60_1}}
	\caption{Исходные данные для сепарации}
	\label{ris:Ex60_1}
\end{figure}

где

$d_{cas}$ - диаметр обсадной колонны, мм

$d_{intake}$ - диаметр приема погружного оборудования, мм

$P_{intake}$ - давление на приеме, атм

$T_{intake}$ - температура на приеме, С

Для вычисления коэффициента естественной сепарации в зависимости от дебита вставьте в ячейку E32 следующую формулу 

{ \small  \texttt{=MF\_ksep\_natural\_d(C32; wc\_; Pintake\_; Tintake\_; Dintake\_; Dcas\_; PVT\_str\_)}}

Для проведения экспериментов по влиянию изменения диаметра обсадной колонны воспользуйтесь в ячейке F32 формулой

{ \small  \texttt{=MF\_ksep\_natural\_d(C32; wc\_; Pintake\_; Tintake\_; Dintake\_; Dcas\_*cf\_dcas\_; PVT\_str\_)}}

При этом в ячейке F30 с помощью коэффициента Вы можете варьировать диаметр обсадной колонны

Для расчета доли газа в газосепараторе применяется функция

{ \small  \texttt{=MF\_gas\_fraction\_d(Pintake\_;Tintake\_;0;PVT\_str\_)*(1-F32)
}}

Коэффициент сепарации газосепаратора

{ \small  \texttt{=MF\_ksep\_gasseparator\_d(gassep\_type;G32;C32)
}}

При этом можно менять тип газосепаратора в ячейке H30

Общий коэффициент сепарации

{ \small  \texttt{=MF\_ksep\_total\_d(E32;H32)
}}

\begin{figure}[h!]
	\center{\includegraphics[width=1\linewidth]{Ex60_2}}
	\caption{Результаты расчета естественной и искусственной сепарации}
	\label{ris:Ex60_2}
\end{figure}

Вопросы к упражнению

\begin{enumerate}
	\item От каких параметров будет зависеть коэффициент сепарации?
	\item Как взаимосвязана естественная и искусственная сепарация? 
\end{enumerate}
