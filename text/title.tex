% Титульный лист (ГОСТ Р 7.0.11-2001, 5.1)
\thispagestyle{empty}
\begin{center}

\end{center}
%
\vspace{0pt plus4fill} %число перед fill = кратность относительно некоторого расстояния fill, кусками которого заполнены пустые места
\IfFileExists{images/logo.pdf}{
  \begin{minipage}[b]{0.5\linewidth}
    \begin{flushleft}
%      \includegraphics[height=3.5cm]{logo}
    \end{flushleft}
  \end{minipage}%
  \begin{minipage}[b]{0.5\linewidth}
    \begin{flushright}
      На правах рукописи\\
%      \textsl {УДК \thesisUdk}
    \end{flushright}
  \end{minipage}
}{
\begin{flushright}
На правах рукописи

%\textsl {УДК \thesisUdk}
\end{flushright}
}
%
\vspace{0pt plus6fill} %число перед fill = кратность относительно некоторого расстояния fill, кусками которого заполнены пустые места
\begin{center}
{\large \thesisTitle}
\end{center}
%
\vspace{0pt plus1fill} %число перед fill = кратность относительно некоторого расстояния fill, кусками которого заполнены пустые места
\begin{center}
\textbf {\large %\MakeUppercase
Unifloc 7 VBA}

\vspace{0pt plus2fill} %число перед fill = кратность относительно некоторого расстояния fill, кусками которого заполнены пустые места
{%\small

}

\vspace{0pt plus2fill} %число перед fill = кратность относительно некоторого расстояния fill, кусками которого заполнены пустые места
\unf


\end{center}
%
\vspace{0pt plus4fill} %число перед fill = кратность относительно некоторого расстояния fill, кусками которого заполнены пустые места
\begin{flushright}
Внимание!!!

\unf{} - переходная версия с большим количеством изменений относительно 7.25

В описании возможно наличие большого числа неточностей. 

Смотрите код или используйте версию 7.25!

\end{flushright}
%
\vspace{0pt plus4fill} %число перед fill = кратность относительно некоторого расстояния fill, кусками которого заполнены пустые места
{\centering Москва 2021\par}

\newpage
История и авторы.

Расчетные модули \unf{} развивались в различных версиях в течении длительного периода времени примерно с 2001 года. Первая версия (условно) - это расчет потенциала добычи нефти для технологического режима добывающих скважин выполненный под руководством Хасанова М.М. Форма технологического режима добычи нефти или форма расчета потенциала уже была разработана и применялось на тот момент. Но она использовала расчет забойного давления по формуле \(P_{intake} = P_{cas}+\rho_{o}g(H_{pump}-H_{dyn})\),  с постоянным значением \(\rho_{o} = 0.86\), что давало большую погрешность на ряде скважин. В первой версии унифлок появился забойного давления по динамическому уровню по оригинальной методике \cite{Khasanov_TR_2006}. Тогда же была сформирована база расчетов PVT корреляций, которая была включена в расчетные модули \cite{Yukos_PVT_2002}. Эта версия широко распространилась и ее можно встретить в различных компаниях. Дальше модули развивались под различные задачи разными коллективами, и единой системы версий не существовало. Поэтому текущая система версий основана на тех проектах, в которых я принимал непосредственное участие. Вторая версия представляет собой набор расчетных модулей для анализа работы фонтанирующих скважин, анализа отжима динамического уровня \cite{Khasanov_depress_test_2010, Khasanov_depress_test_SPE_2010}. Появились расчеты по разным многофазным корреляциям \cite{Khasanov_Unified_SPE_2006,Khabibullin_self_flow_2006}. По названию корреляции - унифицированной \href{http://www.tuffp.utulsa.edu/}{TUFFP} появилось и название unifloc (\textbf{uni}fied \textbf{flo}w \textbf{c}orrelation).  Третья версия появилась примерно тогда же. В ней были попытки расчета скважин с УЭЦН. С ее помощью были подготовлены работы \cite{SPE_117414_2008, SPE_117415_2008, SPE_120628}. Но эта версия расчётных модулей не получила широкого распространения в то время, хотя эти расчётные модули ещё можно найти. К четвёртой версии расчётов можно отнести группу расчётных модулей имеющих общее название - шаблоны применения технологий механизированной добычи. \cite{AL_appl_patt_2007,AL_appl_patt_Vankor_2007}. Некоторые из них до сих пор применяются в компаниях. Пятая версия разрабатывалась уже в компании Газпромнефть. Это в основном расчёты газлифтного фонда скважин. и адаптация расчётных модулей для проведения инженерных расчетов в различных информационных системах \cite{Burtzev_Orenbung_gaslift_2015, offshore_gaslift_2015} . Шестая версия - различные варианты расчёта сделанные на основе предыдущих, но не получившие широкого распространения (расчет динамического уровня по данным эхолокации, расчёта давления в паронагнетательной скважине и тому подобное).  Все эти версии в основном носили прикладной характер и создавались для решения определённых задач. Седьмая, текущая версия информационной системы создавалась для задач обучения, что отличает ее от остальных. Она ориентирована как на обучение студентов ВУЗов (использовалась для проведения курсов в РГУ нефти и газа имени И.М.Губкина, МФТИ, РЭШ), так и на обучение специалистов в ходе коротких курсов повышения квалификации. Эта версия сильно отличается от предыдущих. Исходный код переписан чуть менее чем полностью, проведён рефакторинг методик и алгоритмов, созданы два уровня API - на уровне классов и пользовательских функций, создана и развивается документация. При этом значительная часть функциональности предыдущих версий не реализована. Реализованы только базовые алгоритмы и методики. Большой вклад в развитие текущей версии внесли студенты и аспиранты кафедры РиЭНМ РГУ нефти и газа имени И.М.Губкина. Историю развития проекта можно проследить в репозитории на гитхабе.

Авторы \unf{}
\begin{itemize}
	\item Хабибуллин Ринат
	\item Краснов Виталий
	\item Горидько Кирилл 
	\item Халиков Руслан
	\item Водопьян Алексей
	\item Киян Артем
	\item Кобзарь Олег
	\item Шабонас Артур 
	\item Полешко Михаил
\end{itemize} 
