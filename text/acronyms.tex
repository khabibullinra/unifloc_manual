\chapter*{Единицы измерений} % Заголовок
\addcontentsline{toc}{chapter}{Единицы измерений}  % Добавляем его в оглавление
\noindent

\section*{Давление}
 
 atm, атм "--- физическая атмосфера 
 
 atma, атма "--- абсолютное значение величины в атмосферах
 
 atmg, атми "--- избыточное (измеренное) значение величины в атмосферах. отличается от абсолютной на величину атмосферного давления (1.01325 атма)

\chapter*{Список сокращений и условных обозначений} % Заголовок
\addcontentsline{toc}{chapter}{Список сокращений и условных обозначений}  % Добавляем его в оглавление
\noindent

$\gamma_g$  - \mintinline{vb.net}{gamma_gas} - удельная плотность газа, по воздуху. 

$\gamma_o$  - \mintinline{vb.net}{gamma_oil} - удельная плотность нефти, по воде.

$\gamma_w$  - \mintinline{vb.net}{gamma_wat}- удельная плотность воды, по воде. 

$R_{sb}$ - \mintinline{vb.net}{Rsb_m3m3} газосодержание при давлении насыщения,  $\text{м}^3/\text{м}^3$. 

$R_p$ - \mintinline{vb.net}{Rp_m3m3}. замерной газовый фактор, $\text{м}^3/\text{м}^3$.

$P_b$ - \mintinline{vb.net}{Pb_atma}. давление насыщения, атма.  

$T_{res}$ - \mintinline{vb.net}{Tres_C} пластовая температура, \textcelsius. 

$B_{ob}$ - \mintinline{vb.net}{Bob_m3m3} объёмный коэффициент нефти,  $\text{м}^3/\text{м}^3$. 

$\mu_{ob}$ - \mintinline{vb.net}{Muob_cP}. вязкость нефти при давлении насыщения, сП. 

$Q_{liq}$ - \mintinline{vb.net}{Qliq_scm3day}. дебит жидкости измеренный на поверхности (приведенный к стандартным условиям), м3/сут. 

$f_{w}$ - \mintinline{vb.net}{fw_perc, fw_fr} объёмная обводненность (fraction of water), проценты или доли единиц. 

$f_{g}$ - \mintinline{vb.net}{fg_perc, fg_fr} объёмная доля газа в потоке (fraction of gas), проценты или доли единиц. 

$PI$ - \mintinline{vb.net}{pi_sm3dayatm} - коэффициент продуктивности скважины, $\text{м}^3$/сут/атм

$\rho_{air}$ - \mintinline{vb.net}{rho_air} - плотность воздуха, относительная плотность газа $\gamma_g$ считается по воздуху $\rho_{air} = 1.22$ кг/$\text{м}^3$