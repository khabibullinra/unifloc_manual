%%% Проверка используемого TeX-движка %%%
\newif\ifxetexorluatex   % определяем новый условный оператор (http://tex.stackexchange.com/a/47579)
\ifxetex
    \xetexorluatextrue
\else
    \ifluatex
        \xetexorluatextrue
    \else
        \xetexorluatexfalse
    \fi
\fi


\usepackage{etoolbox}[2015/08/02]               % Для продвинутой проверки разных условий

%%% Поля и разметка страницы %%%
\usepackage{pdflscape}                              % Для включения альбомных страниц
\usepackage{geometry}                               % Для последующего задания полей

%%% Математические пакеты %%%
\usepackage{amsthm,amsmath,amscd}   % Математические дополнения от AMS
\usepackage{amsfonts,amssymb}       % Математические дополнения от AMS
\usepackage{mathtools}              % Добавляет окружение multlined
\usepackage{xfrac}                  % Красивые дроби
\usepackage[
    locale = DE,
    list-separator       = {;\,},
    list-final-separator = {;\,},
    list-pair-separator  = {;\,},
    range-phrase={\text{\ensuremath{-}}},
    % quotient-mode        = fraction, % красивые дроби могут не соответствовать ГОСТ
    fraction-function    = \sfrac,
    separate-uncertainty,
    ]{siunitx}                      % Размерности SI
\sisetup{inter-unit-product = \ensuremath{{}\cdot{}}}

% Кириллица в нумерации subequations
% Для правильной работы требуется выполнение сразу после загрузки пакетов
\patchcmd{\subequations}{\def\theequation{\theparentequation\alph{equation}}}
{\def\theequation{\theparentequation\asbuk{equation}}}
{\typeout{subequations patched}}{\typeout{subequations not patched}}

%%%% Установки для размера шрифта 14 pt %%%%
%% Формирование переменных и констант для сравнения (один раз для всех подключаемых файлов)%%
%% должно располагаться до вызова пакета fontspec или polyglossia, потому что они сбивают его работу
\newlength{\curtextsize}
\newlength{\bigtextsize}
\setlength{\bigtextsize}{13.9pt}

\makeatletter
%\show\f@size                                       % неплохо для отслеживания, но вызывает стопорение процесса, если документ компилируется без команды  -interaction=nonstopmode
\setlength{\curtextsize}{\f@size pt}
\makeatother

%%% Кодировки и шрифты %%%
\ifxetexorluatex
    \PassOptionsToPackage{no-math}{fontspec}        % https://tex.stackexchange.com/a/26295/104425
    \usepackage{polyglossia}[2014/05/21]            % Поддержка многоязычности (fontspec подгружается автоматически)
\else

\fi

%%% Оформление абзацев %%%
\usepackage{indentfirst}                            % Красная строка

%%% Цвета %%%

\usepackage[dvipsnames, table, hyperref]{xcolor} % Совместимо с tikz

%%% Таблицы %%%
\usepackage{longtable,ltcaption} % Длинные таблицы
\usepackage{multirow,makecell}   % Улучшенное форматирование таблиц
\usepackage{tabu, tabulary}      % таблицы с автоматически подбирающейся
                                 % шириной столбцов (tabu обязательно
                                 % до hyperref вызывать)
\usepackage{threeparttable}      % автоматический подгон ширины подписи таблицы

%%% Общее форматирование
\usepackage{soulutf8}                               % Поддержка переносоустойчивых подчёркиваний и зачёркиваний
\usepackage{icomma}                                 % Запятая в десятичных дробях

%%% Оптимизация расстановки переносов и длины последней строки абзаца
\IfFileExists{impnattypo.sty}{% проверка установленности пакета impnattypo
    \ifluatex
        \ifnumequal{\value{draft}}{1}{% Черновик
            \usepackage[hyphenation, lastparline, nosingleletter, homeoarchy,
            rivers, draft]{impnattypo}
        }{% Чистовик
            \usepackage[hyphenation, lastparline, nosingleletter]{impnattypo}
        }
    \else
        \usepackage[hyphenation, lastparline]{impnattypo}
    \fi
}{}

%% Векторная графика

\usepackage{tikz}                   % Продвинутый пакет векторной графики
\usetikzlibrary{chains}             % Для примера tikz рисунка
\usetikzlibrary{shapes.geometric}   % Для примера tikz рисунка
\usetikzlibrary{shapes.symbols}     % Для примера tikz рисунка
\usetikzlibrary{arrows}             % Для примера tikz рисунка

% для раскраски листингов кода
\usepackage{minted}
\usemintedstyle{vs}

\usepackage{csquotes} % biblatex рекомендует его подключать. Пакет для оформления сложных блоков цитирования.
%%% Гиперссылки %%%
\usepackage{hyperref}[2012/11/06]

%%% Изображения %%%
\usepackage{graphicx}[2014/04/25]                   % Подключаем пакет работы с графикой

%%% Счётчики %%%
\usepackage[figure,table]{totalcount}               % Счётчик рисунков и таблиц
\usepackage{totcount}                               % Пакет создания счётчиков на основе последнего номера подсчитываемого элемента (может требовать дважды компилировать документ)
\usepackage{totpages}                               % Счётчик страниц, совместимый с hyperref (ссылается на номер последней страницы). Желательно ставить последним пакетом в преамбуле


% если текущий процесс запущен библиотекой tikz-external, то прекомпиляция должна быть включена
\ifdefined\tikzexternalrealjob
    \setcounter{imgprecompile}{1}
\fi

\ifnumequal{\value{imgprecompile}}{1}{% Только если у нас включена предкомпиляция
    \usetikzlibrary{external}   % подключение возможности предкомпиляции
    \tikzexternalize[prefix=images/cache/] % activate! % здесь можно указать отдельную папку для скомпилированных файлов
    \ifxetex
        \tikzset{external/up to date check={diff}}
    \fi
}{}
